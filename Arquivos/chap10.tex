\chapter{Formas de acesso ao curso}
\label{cap10}

Existem três formas de ingresso aos cursos da UFPE, além da transferência por "força de lei". A primeira, e mais importante, é através do SISU, a segunda é a extra-vestibular; e a terceira através da realização de convênios entre a UFPE e outras instituições, inclusive de fora do país.

O ingresso através do SISU ocorrerá de acordo com o calendário estabelecido pelo MEC, e usará os resultados do ENEM com pesos específicos estabelecidos anualmente pela UFPE em edital próprio.

O Ingresso extra-vestibular é oferecido periodicamente, de acordo com edital próprio da Prograd, através de vagas ociosas nos diversos cursos de graduação em diferentes áreas de conhecimento/formação profissional por meio de transferência interna, transferência externa, reintegração e ingresso em outra habilitação ou outro curso de graduação para diplomados. Para os casos de transferência externa, o candidato deverá já ter cumprido 25\% da carga horária do curso, ou seja, ter concluído os primeiros semestres. Será preciso também comprovar ter menos de 70\% da carga horária a cumprir para conseguir a transferência.

Os convênios entre a UFPE e outras Instituições são conduzidos por uma coordenação específica ligada à Reitoria para o caso dos convênios internacionais e ligada à PROACAD para os casos de convênios nacionais.

Para discentes desvinculados da UFPE, discentes vinculados à outra instituição e discentes já graduados, é possível também realizar matrícula para cursar disciplinas isoladas no Curso de Engenharia Eletrônica. Isto deve ser feito seguindo as regras definidas no edital de matrícula divulgado semestralmente pela Prograd, http://www.ufpe.br/prograd. Não é permitida a matrícula em disciplinas isoladas para discentes com vínculo ativo com a UFPE.

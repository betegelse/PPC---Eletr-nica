\chapter{Histórico da UFPE e do Curso de Engenharia Eletrônica}
\label{cap:um}

A fundação da Universidade do Recife (UR), em 11 de agosto de 1946, dá início à história da Universidade Federal de Pernambuco. A UR foi criada por meio do Decreto-Lei da Presidência da República nº 9.388, de 20 de junho de 1946, e reunia a Faculdade de Direito do Recife, fundada em 1827, a Escola de Engenharia de Pernambuco (1895), a Faculdade de Medicina do Recife (1927), com as escolas anexas de Odontologia (1913) e Farmácia (1903), a Escola de Belas Artes de Pernambuco (1932) e a Faculdade de Filosofia do Recife (1941). Os Institutos de Geociências, Física, e Ciências do Homem, entre outros, foram criados na década de 60.

A Universidade do Recife foi federalizada pela Lei nº 4.759, de 20 de agosto de 1965. Então, a UR passou a integrar o grupo de instituições federais do novo sistema de educação do país, e recebeu a denominação de Universidade Federal de Pernambuco (UFPE), autarquia vinculada ao Ministério da Educação.

Estabelecida em 1975, a atual estrutura de centros e departamentos da UFPE compreende 13 centros acadêmicos, sendo 11 no Recife, um em Caruaru (campus do Agreste, a 140 km do Recife) e outro em Vitória de Santo Antão (a 55 km do Recife). São ofertados atualmente 104 cursos de graduação presenciais, sendo 86 cursos no campus Recife, 12 no campus do Agreste, e 6 no campus de Vitória de Santo Antão, com um total de 28.989 alunos matriculados (dados do semestre 2020.1). Além dos cursos presenciais, a UFPE oferece 5 cursos de graduação a distância, com 449 alunos matriculados. A UFPE oferece ainda 152 cursos de pós-graduação stricto sensu, sendo 74 Mestrados Acadêmicos, 18 Mestrados Profissionais e 54 Doutorados, assim como 22 cursos de pós-graduação lato sensu presenciais (especializações). Há também 362 projetos de extensão voltados para a comunidade.

A estrutura física da UFPE é complementada, no campus Recife, por uma Biblioteca Central, 10 bibliotecas setoriais, o Núcleo de Tecnologia da Informação, a Editora UFPE, o Núcleo de Educação Física e Desportos, o Laboratório de Imunopatologia Keiso-Asami, o Núcleo de Saúde Pública e o Hospital das Clínicas. Na cidade do Recife, encontra-se o Núcleo de Educação Continuada, o Departamento de Extensão Cultural, o Memorial da Universidade de Medicina, o Teatro Joaquim Cardozo e o Núcleo de Rádio e Televisão. Em cidades vizinhas a Recife, duas unidades avançadas de pesquisa completam a estrutura da UFPE: Estação Ecológica Serra dos Cavalos (em Caruaru), e Estação de Itamaracá.

A unidade responsável pela manutenção do Curso de Engenharia Eletrônica na UFPE será o Departamento de Eletrônica e Sistemas — DES. O DES teve origem no antigo Departamento de Engenharia Elétrica (DEE) da UFPE. Em 1950 foi graduada a primeira turma de engenheiros eletricistas formados no DEE da UFPE.

Um programa conjunto com o Departamento de Física nessa época, com o apoio inicial do BNDE/TELEBRÁS e posteriormente da FINEP, fortaleceu o grupo de Eletrônica. Em 1977 foi implementado no DEE o programa de Pós-Graduação com a criação do Mestrado. Foram iniciadas atividades de pesquisa e criadas novas disciplinas nos cursos de graduação e pós-graduação em Arquitetura e Organização de Computadores, Bioeletrônica, Processamento de Sinais de Vídeo, Sistemas de Comunicações, Sistemas de Controle, Reconhecimento de Padrões, Teoria da Informação, Decisão e Planejamento, Fontes não Convencionais de Energia e Dispositivos de Microondas.  Como consequência, a área de Eletrônica e Sistemas, que inicialmente representava uma pequena fração das atividades do DEE, passou a constituir uma parte substancial das atribuições deste Departamento. Isso exigiu das referidas áreas uma maior representatividade, resultando, então, na criação do Departamento de Eletrônica e Sistemas (DES) em outubro de 1979.

Atualmente o DES conta com vinte e nove professores, todos doutores e em regime de Dedicação Exclusiva. Sete professores do DES são bolsistas de produtividade em pesquisa do CNPq. A maioria dos professores participa dos quatro principais grupos de pesquisa do DES credenciados pela UFPE e CNPq: Eletrônica, Engenharia da Informação, Fotônica e Comunicações. Estes grupos mantêm laboratórios e atuam em ensino, pesquisa e extensão nas áreas de controle, processamento de sinais, eletrônica analógica e digital, comunicações móveis, comunicações ópticas, códigos corretores de erros, criptografia, teoria da informação, propagação de ondas eletromagnéticas, antenas, dispositivos de micro-ondas, inteligência artificial, sistemas embarcados e outras. Portanto, a Engenharia Eletrônica constitui-se em uma vocação natural do DES.

Atualmente o DES responde pelos Cursos de Graduação em Engenharia Elétrica/Eletrônica\footnote{A partir de 2010, o curso Engenharia Elétrica/Eletrônica passou a se chamar Engenharia Eletrônica com a criação do perfil 4506 (cópia exata do perfil 4505).
} (conhecido como Engenharia Eletrônica) e Engenharia de Telecomunicações, e contribui fortemente com os Cursos de Graduação em Engenharia Elétrica, Engenharia de Controle e Automação, Engenharia Biomédica e Engenharia da Computação da UFPE.

\chapter{Perfil Profissional do Egresso}
\label{cap5}

O perfil dos egressos do curso de Engenharia Eletrônica da UFPE compreenderá uma sólida formação técnica-científica e profissional geral que capacite-o a absorver e desenvolver novas tecnologias, estimulando a sua atuação crítica e criativa na identificação e resolução de problemas, considerando seus aspectos políticos, econômicos, sociais, ambientais e culturais, com visão ética, humanística e cidadã, em atendimento às demandas da região e da sociedade como um todo. O profissional a ser formado na UFPE em Engenharia Eletrônica terá as seguintes atuações, de acordo com o CONFEA em sua Resolução n.º 218, Art. 9º: “o desempenho das atividades [...] referentes a materiais elétricos e eletrônicos; equipamentos eletrônicos em geral; sistemas de comunicação e telecomunicações; sistemas de medição e controle elétrico e eletrônico; seus serviços afins e correlatos” (Anexo 1).

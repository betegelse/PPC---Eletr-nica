\chapter{Objetivos}
\label{cap4}

O objetivo do curso de Engenharia Eletrônica é formar profissionais com uma sólida capacidade técnica aliada a uma visão ética, ambientalista e humanista para atender à demanda tecnológica e científica do mercado local e global, incluindo a carreira acadêmica. 

Os objetivos específicos que se pretende alcançar, em consonância com as diretrizes curriculares (Anexo 1), são:
\begin{enumerate}
	\item[$\bullet$]capacitar o discente a:
\begin{itemize}
	\item[-] elaborar, executar e analisar projetos técnicos e científicos;
	\item[-]acompanhar as evoluções tecnológicas da Engenharia Eletrônica;
	\item[-]desenvolver pesquisas em eletrônica;
	\item[-]atuar administrativamente no desempenho de funções relacionadas à Engenharia Eletrônica.
\end{itemize}
	\item[$\bullet$]oferecer um ensino centrado no discente e voltado para os resultados do aprendizado;
	\item[$\bullet$]enfatizar a solução de problemas de engenharia;
	\item[$\bullet$]formar profissionais adaptáveis às rápidas evoluções tecnológicas;
	\item[$\bullet$]oportunizar uma sólida formação geral;
	\item[$\bullet$]proporcionar uma articulação com a pós-graduação com ênfase na inovação;
	\item[$\bullet$]incentivar a interdisciplinaridade.
\end{enumerate}

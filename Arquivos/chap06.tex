\chapter{Campo de atuação do profissional}
\label{cap6}

O Engenheiro Eletrônico da UFPE será um profissional de formação generalista, que atua na área de materiais eletro-eletrônicos; sistemas de medição e de controle eletro-eletrônico; desenvolvimento de sistemas, produtos e equipamentos eletrônicos, sistemas embarcados, conversores, equipamentos biomédicos e informática médica. Estuda, projeta e especifica materiais, componentes, dispositivos e equipamentos eletro-eletrônicos, eletromecânicos, magnéticos, ópticos, de instrumentação, sensores e atuadores de transmissão e recepção de dados, de áudio/vídeo, de segurança patrimonial e de eletrônica embarcada. Planeja, projeta, instala, opera e mantém sistemas e instalações eletrônicas, equipamentos, dispositivos e componentes odonto-médico-hospitalares e de instrumentação biomédica, sistemas de medição e instrumentação eletro-eletrônica, de acionamentos de máquinas, de controle eletrônico e de automação, e de sistemas eletrônicos embarcados. Coordena e supervisiona equipes de trabalho, realiza estudos de viabilidade técnico-econômica, executa e fiscaliza obras e serviços técnicos; e efetua vistorias, perícias e avaliações, emitindo laudos e pareceres. Em suas atividades, considera a ética, a segurança, a legislação e os impactos ambientais.

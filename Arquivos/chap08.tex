\chapter{Metodologia de ensino do curso}
\label{cap8}

Durante todo o curso, o professor acompanhará a aprendizagem do aluno, observando seu desenvolvimento real (o que ele já conhece) e seu desenvolvimento potencial (o que ele pode realizar com ajuda). Essa ajuda será mediada\footnote[2]{Baseando-se na teoria de Vygotsky, aprendizagem mediada é aquele que “depende de duas pessoas, uma mais bem informada do que a outra, possibilitando uma mediação social na experiência do aprender, a fim de que o menos habilitado se torne progressivamente capaz, [...].” LINHARES, Maria Beatriz M.; ESCOLANO,  ngela C. M.; ENUMO, Sônia R.F.; (org.) Avaliação Assistida: fundamentos, procedimentos e aplicabilidade. S.P.: Casa do Psicólogo, 2006. p.17.} pelo professor, durante as aulas e em atendimento individualizado; ou pelo monitor, durante as aulas e no contra-turno, em caso das disciplinas obrigatórias com turmas numerosas e que tenham, também, carga horária prática. A diferença entre o que o aluno já sabe e o que ele pode vir a saber (zona de desenvolvimento proximal) será motivo da ação-reflexão-ação da práxis docente\footnote[3]{“O ideal é que a aula seja reflexão que veio da ação e que leva para a ação. [...] que sugira essa aplicação na realidade. E isso pode ser feito em qualquer matéria na medida em que o professor for realmente professor-educador para a liberdade [...]”. FREIRE, Paulo; SANTOS, Vlademir; VANNUCCHI, Aldo (org); S.P.: Loyola, 2003. p.26.} através da avaliação diagnóstica. 

Para as disciplinas obrigatórias, além das aulas expositivas dialogadas, serão utilizadas as técnicas de seminários, resolução de lista de exercícios e realização de projeto final de disciplina. Em algumas aulas práticas, além da resolução de problemas e confecção de relatórios, também haverá, por parte do aluno, a preparação prévia para as práticas. Essa preparação obrigatória consiste em desenvolver um projeto\footnote[4]{Segundo Freire, atividades de projeto contribuem para uma pedagogia da autonomia [que] tem de estar centrada em experiências estimuladoras da decisão e da responsabilidade, vale dizer, em experiências respeitosas da liberdade (2010, p. 107).}, que será implantado com a mediação do professor. 

O curso tem incentivado a aplicação de metodologias ativas no ensino. Por exemplo, várias disciplinas têm projetos práticos utilizando modelagem, simulação e implementação de dispositivos que são encontrados com frequência em aplicações reais. Nesses projetos, é dada ênfase a técnicas com grande participação do aluno no processo de aprendizado. Nelas, o discente é incentivado a usar o conhecimento assimilado nas aulas teóricas e a sua criatividade para projetar, fabricar, medir e testar sistemas eletrônicos precursores daqueles que ele encontrará em sua vida profissional.

Uma expansão desta técnica está sendo estudada para aplicação no curso, onde um projeto será usado transversalmente ao longo de várias disciplinas, permitindo aos discentes estabelecer as relações entre os conteúdos dessas disciplinas e motivando-os a adquirir o conhecimento de disciplinas básicas que serão usados depois nas disciplinas mais avançadas e em aplicações do mundo real.

Uma outra técnica já aplicada em algumas disciplinas do curso é a Sala Invertida, onde o conteúdo a ser estudado é passado para o discente estudar antes da aula, e durante as aulas atividades são desenvolvidas com a participação ativa dos discentes para construção e desenvolvimento do conhecimento.

Espera-se que técnicas como essas promovam um maior engajamento dos alunos no curso, reduzindo suas taxas de evasão e retenção.

Junto com a bibliografia básica e a complementar na biblioteca setorial do Centro de Tecnologia e Geociências, também ficará disponibilizado, para download, o material utilizado nas aulas dos professores (slides, apostilas, etc) através da Internet. 

A integração com a pesquisa, será realizada por meio do incentivo e creditação, como atividade complementar, da participação dos alunos em atividades de pesquisa e desenvolvimento em grupos de pesquisa do DES, seja em programa de iniciação científica, na participação em projetos de P\&D financiados por empresas, ou em atividades e eventos promovidos pelo Programa de Pós-Graduação em Engenharia Elétrica da UFPE.

Já no que diz respeito à acessibilidade metodológica, a UFPE vem desempenhando ações efetivas para a garantia da acessibilidade que repercutem nas atividades de ensino. Como exemplo mais recente, pode-se citar a introdução de projetores multimídias interativos, em que é possível fazer modificação dos slides em projeção, bem como gravar o que está sendo feito. Certamente, alunos com dificuldades ou limitações podem usufruir de melhores condições para aprendizagem uma vez que podem revisar tudo o que é dito em sala na comodidade de suas casas.

Outros aspectos de acessibilidade, como a comunicacional, instrumental, programática, e atitudinal, vem sendo melhorados através do uso de ferramentas de comunicação como as redes sociais, o uso de plataformas computacionais de software livre, o acesso à portais informatizados com informações e normas institucionais, e especialmente com a conscientização da necessidade de tratamento diferenciado àqueles que apresentam dificuldades ou limitações. O tratamento diferenciado aos alunos com dificuldades ou transtornos funcionais específicos da aprendizagem ou superdotação/altas habilidades específicas, oriundas de algum tipo de deficiência, incluindo o autismo conforme a Lei 12.764 de 2012, transtorno global do desenvolvimento ou limitações especiais se materializa, por exemplo, na flexibilização do tipo e tempo de avaliação, no maior tempo de atenção por parte de docentes e monitores dentro e fora de sala de aula, na indicação de material didático adicional diferenciado e adaptado às necessidades.

Por fim, é importante mencionar que, conforme regulamentado pela Resolução No. 13/2016, do Conselho Coordenador de Ensino, Pesquisa e Extensão, o uso da metodologia de ensino a distância será adotado em até 20\% da carga horária do curso. As disciplinas que poderão adotar tal metodologia dependerão de aprovação do Colegiado do curso. 


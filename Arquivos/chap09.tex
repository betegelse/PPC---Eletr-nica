\chapter{Sistemática de avaliação}
\label{cap9}

A UFPE como um todo está em fase de renovação de seu sistema de avaliação, buscando implementar neste uma avaliação que observe não só o aprendizado do aluno como também a sua opinião quanto às práticas pedagógicas adotadas na Universidade. Ainda, está em processo de institucionalização o uso dos resultados do ENADE (prova e questionário) objetivando um melhor aproveitamento e aprimoramento de todo o processo.

Hoje, a avaliação da aprendizagem da UFPE é regida pela Resolução 04/1994 do CCEPE (Conselho Coordenador de Ensino, Pesquisa e Extensão), de 23 de dezembro de 1994. Esta resolução determina a aprovação por média, aprovação, reprovação e reprovação por falta. Regula ainda o sistema de revisão de prova, de realização de segunda chamada entre outras especificidades. O Sistema Integrado de Atividades Acadêmicas da Universidade, o SigaA, garante o cumprimento desta Resolução, garantindo ainda ao aluno a privacidade dos seus resultados.

A Resolução abrange aspectos de:
\begin{enumerate}
\item[$\bullet$] Frequência: considera-se reprovado o aluno que não tiver comprovada sua participação em pelo menos 75\% (setenta e cinco por cento) das aulas teóricas ou práticas computadas separadamente, ou ao mesmo percentual de avaliações parciais de aproveitamento escolar.
\item[$\bullet$] Aproveitamento: avalia-se o aluno em termos de aproveitamento da disciplina ao longo do período letivo, mediante verificações parciais (pelo menos duas), sob forma de provas escritas, orais ou práticas, trabalhos escritos, seminários, e outros. Além disso, ao fim do período letivo, depois de cumprido o programa da disciplina, é realizada uma outra avaliação mediante verificação do aproveitamento de seu conteúdo total sob a forma de exame final. As avaliações de aproveitamento serão expressas em graus numéricos de 0,0 (zero) a 10,0 (dez).
\item[$\bullet$] O aluno que comprovar o mínimo de frequência (75\%) e obtiver uma média parcial (referentes às suas notas das avaliações parciais) igual ou superior a 7,0 (sete) será considerado aprovado na disciplina com dispensa do exame final, tendo registrada a situação final de APROVADO POR MÉDIA em seu histórico escolar, e a sua Média Final será igual à Média Parcial.
\item[$\bullet$] Comprovado o mínimo de frequência (75\%) o aluno será considerado APROVADO na disciplina se obtiver simultaneamente:
\begin{itemize}
	\item[-] Média parcial e nota do exame final não inferiores a 3,0 (três);
    \item[-] Média final (consistindo da média aritmética da média parcial e da nota do exame final) não inferior a 5,0 (cinco)
\end{itemize}
\item[$\bullet$] Ficará impedido de prestar exame final o aluno que não obtiver, no mínimo, 75\% (setenta e cinco por cento) de frequência na disciplina, e/ou não obtiver, no mínimo, 3 (três) como média das duas notas parciais.
\begin{itemize}
	\item[-] Consta ainda no estatuto da Universidade Federal de Pernambuco de 1968:
    \item[-] Será considerado reprovado, em cada disciplina, o aluno que não alcançar, nos trabalhos e exames escolares, as notas mínimas estabelecidas, ou que deixar de comparecer ao mínimo fixado de aulas e trabalhos escolares, vedado o abono de faltas.
\end{itemize}
\end{enumerate}

Terão critérios especiais de avaliação as disciplinas abaixo discriminadas:
\begin{enumerate}
\item[$\bullet$]Estágio Curricular - será observado o que estabelece a Resolução nº. 02/85 do CCEPE;
\item[$\bullet$]Disciplinas que envolvam elaboração de projetos, monografias, trabalho de graduação ou similares, terão critérios de avaliação definidos pelos respectivos Colegiados do Curso.
\end{enumerate}

Poderá ser concedida 2ª chamada exclusivamente para exame final ou para uma avaliação parcial especificada no plano de ensino da disciplina. Ao aluno será permitido requerer até duas revisões de julgamento de uma prova ou trabalho escrito, por meio de pedido encaminhado ao coordenador do curso.

Em todo caso, os diversos procedimentos e práticas relacionados à sistemática de avaliação, como resoluções de listas de exercícios, revisões e as notas propriamente ditas do aluno, devem servir para indicar se existe ou não necessidade de mudança na abordagem, na metodologia ou nas técnicas utilizadas (avaliação diagnóstica). 

A avaliação formativa (avaliação de conteúdo) será flexibilizada em situações de estágio, participação em conferências, intercâmbios etc; sendo, também, revistas suas circunstâncias, caso tenham sido desfavoráveis ao processo, a exemplo de provas muito longas num tempo muito curto. 

De modo mais específico, de acordo com o Projeto Político Pedagógico Institucional (PPPI) 2007 da UFPE, assumimos a perspectiva da avaliação formativa assinalada naquele documento, 

“Na qual o interesse é voltado para o que foi aprendido, o que permite a função reguladora de ajustes à aprendizagem e ao ensino, desenvolvendo o sentido de autonomia e em direção a uma estrutura personalizada e acompanhada das aprendizagens” (p.58-59).

Essa concepção de avaliação é realizada durante todo o semestre letivo, de modo que possa ser verificado se os discentes dominam as etapas gradativa e hierarquicamente do conhecimento, sendo este desdobrado em objetivos, previamente definidos pelo docente, por ocasião da elaboração do plano de ensino do componente curricular a ser ministrado.

Na perspectiva avaliativa colocada por Hoffman (2005, p.129)\footnote[5]{HOFFMANN, J. M. L. Avaliação mediadora: uma prática em construção da pré-escola à universidade. 24. ed. Porto Alegre: Mediação, 2005.}, em uma experiência no Ensino Superior, destacam-se algumas linhas mestras delineadas pela autora:
\begin{enumerate}
\item[$\bullet$]Oportunizem aos alunos muitos momentos para que estes possam expressar suas ideias, retomar dificuldades referentes aos conteúdos trabalhados no início e desenvolvidos ao longo do semestre;
\item[$\bullet$]Garantam a realização de muitas tarefas em grupos, a fim de que os alunos, entre si, se auxiliem nas dificuldades, sem com isso, o professor deixar de acompanhar, individualmente, o aluno, a partir de tarefas avaliativas individuais em todas as etapas do processo;
\item[$\bullet$]Em lugar de simplesmente marcar “certo” e “errado”, ou, textualmente, fazer comentários irônicos, de supremacia e de descrédito, o docente possa fazer anotações significativas para si e para o aluno, apontando-lhe soluções equivocadas e possibilitando aprimoramento em suas resoluções;
\item[$\bullet$]Proporcionem atividades em espiral, ou seja, tarefas relacionadas às anteriores, num processo de complexidade e gradação coerentes às descobertas feitas pelos alunos, às dificuldades feitas por eles, ao desenvolvimento do conteúdo;
\item[$\bullet$]Convertam a tradicional rotina de atribuir conceitos classificatórios às tarefas, calculando médias de desempenho final, em tomada de decisão do professor com base nos registros feitos sobre a evolução dos alunos nas diferentes etapas do processo, tornando o aluno comprometido com tal processo.
\end{enumerate}
Desdobrando essas linhas mestras em instrumentos mais explícitos e específicos de avaliação, neste Projeto Pedagógico de Curso serão utilizadas várias técnicas e instrumentos de avaliação, listados a seguir:
\begin{enumerate}
\item[$\bullet$] Artigos e relatos de experiência;
\item[$\bullet$] Estudos de caso;
\item[$\bullet$] Participação em sala de aula;
\item[$\bullet$] Projetos de pesquisa;
\item[$\bullet$] Projetos executivos;
\item[$\bullet$] Provas práticas;
\item[$\bullet$] Provas teóricas;
\item[$\bullet$] Provas teórico-práticas;
\item[$\bullet$] Relatórios de execução.
\item[$\bullet$] Relatórios de pesquisa;
\item[$\bullet$] Seminários temáticos;
\item[$\bullet$] Trabalhos teóricos;
\item[$\bullet$] Tutoria e orientação;
\end{enumerate}

Registra-se ainda que tais instrumentos de avaliação podem ser periodicamente discutidos pelo Colegiado do Curso e pelo Núcleo Docente Estruturante, com a finalidade de aprimorar e redimensionar as práticas desenvolvidas em sala de aula. Coloca-se ainda que outros instrumentos serão utilizados, sempre que necessário, para adequar as estratégias que surgirem na vigência deste PPC decorrentes das práticas pedagógicas vivenciadas ao longo dos componentes curriculares.

Com relação à acessibilidade do processo avaliativo é notório que cada caso deve ser tratado com toda a devida atenção devido às particularidades de cada indivíduo. Por exemplo, alunos com dificuldades visuais podem ser atendidos com provas com letras maiores ou provas em Braille, dependendo do grau da limitação. O NACE é o setor responsável da universidade para pensar e delimitar estratégias viáveis para cada deficiência encontrada. 

Além da avaliação da aprendizagem, existem outros métodos de avaliação destinados à melhoria do curso e da instituição como um todo. Ao docente, por exemplo, cabe a autoavaliação e a avaliação da infraestrutura da universidade. Já ao discente cabe a avaliação da estrutura da universidade bem como a avaliação dos docentes. A avaliação do docente pelo discente, realizada periodicamente através do SigaA, é um instrumento que permite aos docentes compreender como os alunos enxergam a execução do seu trabalho. Neste sistema informatizado para avaliação do docente pelos discentes, os alunos avaliam o docente de cada disciplina sob vários aspectos, incluindo sua metodologia, capacidade e clareza de comunicação, relacionamento com os alunos, assiduidade e pontualidade nas aulas, capacidade de avaliação e atribuição de notas, cumprimento do plano de ensino, capacidade de motivar os alunos para o aprendizado do conteúdo da disciplina, dentre outros aspectos. Os alunos respondem a um questionário online no sistema SigaA de forma que a sua identidade não é conhecida pelos docentes. Os docentes recebem ao final do período letivo a avaliação na forma de um histograma indicando a distribuição das respostas dos alunos, e uma nota entre 0 e 10 é atribuída ao docente. Tal avaliação também é considerada para a progressão dos docentes em sua carreira. Por fim, a avaliação do curso é realizada pelo Colegiado e NDE através de consultas regulares e diálogo com os discentes e docentes do curso.

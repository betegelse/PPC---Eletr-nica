\chapter{Competências, atitudes e habilidades}
\label{cap7}

Com a sempre presente tendência da tecnologia de ser cada vez mais aplicada aos mais diversos produtos e sistemas, o Engenheiro Eletrônico se encontra no centro de uma grande revolução tecnológica com o potencial para prover profundos benefícios para a sociedade. Criação de sistemas de controle eletrônico, instrumentos de telemetria, processamento digital e analógico, desenvolvimento de computadores, robótica e soluções eletrônicas para a indústria são algumas das áreas com grandes expectativas para o futuro. Assim, para ser capaz de trabalhar com sucesso neste ambiente, não basta para o Engenheiro Eletrônico dominar apenas o conhecimento técnico, é preciso também que ele domine várias habilidades gerais. A formação do Engenheiro Eletrônico terá por objetivo dotar o profissional dos conhecimentos requeridos para o exercício das seguintes competências e habilidades gerais:
\begin{enumerate}
\item[$\bullet$] aplicar conhecimentos matemáticos, científicos, tecnológicos e instrumentais à engenharia elétrica / eletrônica;
\item[$\bullet$] projetar e conduzir experimentos e interpretar resultados;
\item[$\bullet$] conceber, projetar e analisar sistemas, produtos e processos;
\item[$\bullet$] planejar, supervisionar, elaborar e coordenar projetos e serviços de engenharia elétrica/eletrônica;
\item[$\bullet$] identificar, formular e resolver problemas de engenharia elétrica/eletrônica;
\item[$\bullet$] desenvolver e/ou utilizar novas ferramentas e técnicas;
\item[$\bullet$] supervisionar a operação e a manutenção de sistemas elétricos/eletrônicos;
\item[$\bullet$] avaliar criticamente a operação e a manutenção de sistemas elétricos/eletrônicos;
\item[$\bullet$] comunicar-se eficientemente nas formas escrita, oral e gráfica;
\item[$\bullet$] atuar em equipes multidisciplinares;
\item[$\bullet$] compreender e aplicar a ética e responsabilidade;
\item[$\bullet$] avaliar o impacto das atividades da engenharia no contexto social e ambiental;
\item[$\bullet$] avaliar a viabilidade econômica de projetos de engenharia elétrica/eletrônica;
\item[$\bullet$] assumir a postura de buscar, permanentemente, a atualização profissional.
\end{enumerate}
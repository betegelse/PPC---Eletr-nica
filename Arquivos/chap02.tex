\chapter{Justificativa}
\label{cap:dois}

A presença da tecnologia na sociedade humana em todo o mundo vem aumentando ano a ano, expandindo a setores antes alheios a ela. Seja para tornar sistemas e processos mais eficientes, seja para proporcionar distração a indivíduos, a sua influência é sentida cada vez mais profundamente. Na automação de sistemas, na comunicação de informações e no desenvolvimento de novos equipamentos, o impacto do uso da tecnologia vem sendo impressionante.

Às áreas tradicionais de eletrônica e telecomunicações, novos campos de atuação vêm se somando a cada ano. Desenvolvimento de instrumentos biomédicos, implementação de redes de sensores e aplicações em fontes alternativas de energia são fortes exemplos deste fenômeno.

Por outro lado, o estado de Pernambuco vem recebendo investimentos vultosos para o desenvolvimento do seu parque tecnológico, e várias indústrias dos mais diversos setores têm se instalado na região, em particular em SUAPE. Como toda indústria moderna, independente do setor, utiliza maquinaria automatizada, esta expansão do parque industrial vem aumentando a demanda por serviços técnicos especializados na área de eletrônica. Além disso, no estado também existem diversas empresas de desenvolvimento de equipamentos eletrônicos, empresas de desenvolvimento de softwares e empresas de prestação de serviços que necessitam de engenheiros eletrônicos capacitados para realizar os seus serviços de forma eficiente, competente e segura.

O engenheiro eletrônico deverá então ser capaz de estar envolvido neste processo através do projeto, desenvolvimento, testes e implantação de dispositivos, sensores e sistemas eletrônicos, assim como da prestação de serviços especializados e da manutenção e operação de sistemas. Além disso, o engenheiro hoje em dia é solicitado a considerar questões sociais e ambientais, valorizando assim o ser humano e o meio ambiente. Ele deve entender a complexidade dos problemas ambientais e, em consequência, a necessidade de se desenvolver o senso crítico e as habilidades necessárias para resolver estes problemas e construir sociedades socialmente justas, sustentáveis e ecologicamente equilibradas.

A sua formação, portanto, é de extrema importância para o desenvolvimento da região. Nesse contexto, o curso de Engenharia Eletrônica da Universidade Federal de Pernambuco, além de propiciar esta formação, busca desenvolver o trabalho de pesquisa e investigação científica, integrar os conhecimentos profissionais com a estrutura intelectual de cada geração, estimular o conhecimento dos problemas atuais mundiais, nacionais e regionais, ensinar uma capacidade de decisão crítica para destacar os valores éticos e a responsabilidade social, fomentar e inserir o futuro profissional na sociedade em que vive, apresentando-lhes os desafios a serem enfrentados e as suas responsabilidades com o desenvolvimento social, condição necessária para a formação de um profissional ético e um cidadão integral.

A elaboração do presente documento se justifica, também, pela necessidade de resposta a uma demanda natural de áreas como a Engenharia: a constante atualização dos instrumentos relacionados às práticas educacionais nesse contexto, em função de sua rápida e dinâmica evolução sob diversos aspectos. A proposição de um novo projeto pedagógico, associado a um perfil curricular moderno e sincronizado principalmente com o mercado, vem preencher diversas lacunas surgidas ao longo dos últimos anos, conforme delineado nas seções que se seguem.

Por último, este PPC visa adequar o Curso de Engenharia Eletrônica a duas novas resoluções para o ensino superior. A primeira é a Resolução n° 02, de 24 de abril de 2019, do Conselho Nacional de Educação/Câmara de Educação Superior (CNE/CES), que estabelece as novas Diretrizes Curriculares Nacionais do Curso de Graduação em Engenharia. A segunda é a Resolução nº 7, de 18 de dezembro de 2018, do Conselho Nacional de Educação /Câmara de Educação Superior (CNE/CES), que estabelece as diretrizes para a extensão na Educação Superior brasileira e regulamenta a reserva mínima de dez por cento do total de créditos exigidos para a graduação em programas e projetos de extensão universitária, a chamada curricularização da extensão.

\chapter{Marco Teórico}
\label{cap:tres}

O presente projeto pedagógico fundamenta-se na concepção epistemológica de que o “Engenheiro”, sendo criador e aplicador das mais diferentes tecnologias para o benefício da sociedade, é o elemento principal que poderá contribuir como profissional e cidadão para a solução de problemas relacionados à Elétrica / Eletrônica que afligem a coletividade. É um pressuposto que este profissional tenha a capacidade de assimilar outros conhecimentos que o tornem capaz de considerar o ser humano como elemento central de todas as suas atenções, modificando aqueles costumes e culturas que contrariem a necessidade de preservação, comunicação e bem estar de seus semelhantes.

A concepção do currículo do curso de Engenharia Eletrônica e as diretrizes do seu processo adaptativo/evolutivo têm como pilar fundamental a aplicação metodologias de ensino/aprendizagem que promovam a construção do saber crítico e reflexivo. Os componentes curriculares devem proporcionar meios para construção de um novo conceito ou consolidação de um conceito objeto do estudo, com espaço para construção coletiva e participativa.  A metodologia de aprendizagem também deve ser aprimorada a partir da autoavaliação contínua do curso.

Em adição a preocupação de desenvolvimento de uma base sólida do estudante, o curso não descuida dos aspectos relativos aos valores éticos e morais, prevendo dentro o componente curricular obrigatório “sociologia e meio ambiente” a discussão de temáticas importantes para formação do cidadão em suas dimensões mais amplas, como fatores culturais sociais, cultura, interação social, grupos sociais,  normas sociais e Relações Étnico raciais os quais são essenciais para a formação do cidadão consciente de seu papel na sociedade, bem como para a construção de consciência livre de preconceito e com entendimento das particularidades culturais, devido ao conhecimento da história dos diferentes grupos culturais, incluindo a história da cultura Afro-brasileira.

O curso favorece ao desenvolvimento do espírito empreendedor presente em cada aluno, bem como permite a conscientização para o uso equilibrado dos elementos presentes nos sistemas produtivos, em favor do meio ambiente e de uma cultura que promova a sustentabilidade com os olhos voltados para as especificidades regionais.

Oportunidades de aprimoramento da formação acadêmica são proporcionadas durante todo o trajeto dos estudantes no curso, os quais podem participar de iniciação científicas, de projetos de pesquisas, de monitorias, bem como de programas de intercâmbio. Tais ações são incentivadas, bem como instruídas pela coordenação do curso.

Além do ensino e da pesquisa, os alunos do curso têm participação em atividades de extensão, as quais são oferecidas por órgão específico dentro da universidade. É importante destacar que as atividades de extensão se constituem num importante e eficaz instrumento institucional que promove a troca de saberes e a integração com a sociedade. Além disso, ao mesmo tempo em que beneficia a população, contribuindo para a melhoria da qualidade de vida, inclusão sócio-produtiva e defesa do meio ambiente, as ações extensionistas – que incluem atividades técnicas, científicas, culturais e artísticas – propiciam ao estudante a oportunidade para um aprendizado teórico-prático contextualizado, desenvolvimento cultural, responsabilidade social e formação da cidadania.

Por tudo que foi colocado, o currículo ora proposto pelo curso estabelece a integração entre ensino, pesquisa e extensão universitária como metas constantes e integradas. 
